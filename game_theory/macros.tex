\newcommand{\ra}[1]{\renewcommand{\arraystretch}{#1}}

\newtheorem{thm}{Theorem}[section]
\newtheorem{prop}[thm]{Proposition}
\newtheorem{lem}[thm]{Lemma}
\newtheorem{cor}[thm]{Corollary}
\newtheorem{defn}[thm]{Definition}
\newtheorem{rem}[thm]{Remark}
\numberwithin{equation}{section}
\newtheorem{example}[thm]{Example}
\newcommand{\homework}[6]{
   \pagestyle{myheadings}
   \thispagestyle{plain}
   \newpage
   \setcounter{page}{1}
   \noindent
   \begin{center}
   \framebox{
      \vbox{\vspace{2mm}
    \hbox to 6.28in { {\bf DCS 2019:~Social \& Economic Networks \hfill {\small (#2)}} }
       \vspace{6mm}
       \hbox to 6.28in { {\Large \hfill #1  \hfill} }
       \vspace{6mm}
       \hbox to 6.28in { {\it Instructor: {\rm #3} \hfill Name: {\rm #5}, Netid: {\rm #6}} }
       %\hbox to 6.28in { {\it TA: #4  \hfill #6}}
      \vspace{2mm}}
   }
   \end{center}
   \markboth{#5 -- #1}{#5 -- #1}
   \vspace*{4mm}
}

\newcommand{\topic}[2]{~\\\fbox{\textbf{Topic #1}}\hfill (#2 hours)\newline\newline}
\newcommand{\subtopic}[1]{~\newline\textbf{(#1)}}
\newcommand{\D}{\mathcal{D}}
\newcommand{\Hy}{\mathcal{H}}
\newcommand{\VS}{\textrm{VS}}
\newcommand{\solution}{~\newline\textbf{\textit{(Solution)}} }

\newcommand{\bbF}{\mathbb{F}}
\newcommand{\bbX}{\mathbb{X}}
\newcommand{\bI}{\mathbf{I}}
\newcommand{\bX}{\mathbf{X}}
\newcommand{\bY}{\mathbf{Y}}
\newcommand{\bepsilon}{\boldsymbol{\epsilon}}
\newcommand{\balpha}{\boldsymbol{\alpha}}
\newcommand{\bbeta}{\boldsymbol{\beta}}
\newcommand{\0}{\mathbf{0}}


% NOTATIONS MATHEMATIQUES
\newcommand{\Rset}{\mathbb{R}}
\newcommand{\Rbb}{\mathbb{R}}
\newcommand{\Nset}{\mathbb{N}}
\newcommand{\ud}{\mathrm{d}}
\newcommand{\var}{\mathbb{V}}
\newcommand{\cov}{\mathrm{cov}}
\newcommand{\E}{\mathbb{E}}
\newcommand{\Beta}{B}
\newcommand{\Nbb}{\mathbb{N}}

\newcommand{\projorth}[2]{\text{proj}^{\bot}_{#2}(#1)}
\newcommand{\proj}[2]{\text{proj}_{#2}(#1)}
\newcommand{\argmax}{\mathop{\mathrm{arg\ max}}}
\newcommand{\argmin}{\mathop{\mathrm{arg\ min}}}
\newcommand{\minimize}{\mathop{\mathrm{minimize}}}
\newcommand{\maximize}{\mathop{\mathrm{maximize}}}
\newcommand{\set}[1]{\left\{#1\right\}}
% definitions related to groups
\newcommand{\norm}[2][]{\left|\left|#2\right|\right|_{#1}}
\newcommand{\group}[1][k]{{\mathcal G}_{#1}}
\newcommand{\positive}{{\mathcal P}}
\newcommand{\negative}{{\mathcal N}}
\newcommand{\zero}{{\mathcal Z}}
%\renewcommand{\active}[1][k]{{\mathcal A}_{#1}}
\newcommand{\1}{\mathbf{1}}

\newcommand{\tr}{\mathrm{tr}}
\newcommand{\trace}[1]{\mathrm{trace}{\left(#1\right)}}
\newcommand{\vect}{\mathrm{vec}}
\newcommand{\sign}{\mathrm{sign}}
\newcommand{\err}{\mathrm{err}}
\newcommand{\weights}{\mathbf{w}}
\newcommand{\supp}{\mathcal{A}}
\newcommand{\prob}{\mathbb{P}}
\renewcommand{\P}{\mathbb{P}}

%definitions related to convergences
\newcommand{\inprob}{\overset{P}{\longrightarrow}}
\newcommand{\inlaw}{\overset{D}{\longrightarrow}}

\newcommand{\C}{\texttt}
\newcommand{\R}{\C{R}}
\newcommand{\easy}{\mbox{\large\fontencoding{U}\fontfamily{wasy}\selectfont\char44}}
\newcommand{\medium}{\mbox{\large\fontencoding{U}\fontfamily{wasy}\selectfont\char47}}
\newcommand{\hard}{\raisebox{2pt}{\footnotesize\fontencoding{U}\fontfamily{futs}\selectfont\char77}}

\newcommand{\Easy}{\mbox{\Huge\fontencoding{U}\fontfamily{wasy}\selectfont\char44}}
\newcommand{\Medium}{\mbox{\Huge\fontencoding{U}\fontfamily{wasy}\selectfont\char47}}
\newcommand{\Hard}{\raisebox{3pt}{\huge\fontencoding{U}\fontfamily{futs}\selectfont\char77}}

\newcommand{\red}[1]{\textcolor{red}{#1}}
\newcommand{\green}[1]{\textcolor{green!50!black}{#1}}

\newcommand\indep{\protect\mathpalette{\protect\independenT}{\perp}}
\def\independenT#1#2{\mathrel{\rlap{$#1#2$}\mkern2mu{#1#2}}}

%-------------------------------------------------------------------------%
% Definitions
%-------------------------------------------------------------------------%
\def\Argmin{\mathop{\mbox{\rm argmin}}}
\def\Argmax{\mathop{\mbox{\rm argmax}}}

\newcommand{\hatbeta}{\hat{\beta}}
\newcommand{\hatbbeta}{\,\hat{\!\bbeta}}
\newcommand{\hatbetalasso}{\,\hat{\!\beta}^{\mathrm{lasso}}}
\newcommand{\hatbetacoop}{\,\hat{\!\beta}^{\mathrm{coop}}}
\newcommand{\hatbetagroup}{\,\hat{\!\beta}^{\mathrm{group}}}
\newcommand{\hatbetaspgroup}{\,\hat{\!\beta}^{\mathrm{spgroup}}}
\newcommand{\hatbetaridge}{\,\hat{\!\beta}^{\mathrm{ridge}}}
\newcommand{\hatbetaols}{\,\hat{\!\beta}^{\mathrm{ols}}}
\newcommand{\hatbbetacoop}{\,\hat{\!\bbeta}^{\mathrm{coop}}}
\newcommand{\hatbbetagroup}{\,\hat{\!\bbeta}^{\mathrm{group}}}
\newcommand{\hatbbetaridge}{\,\hat{\!\bbeta}^{\mathrm{ridge}}}
\newcommand{\hatbbetalasso}{\,\hat{\!\bbeta}^{\mathrm{lasso}}}
\newcommand{\hatbbetaols}{\,\hat{\!\bbeta}^{\mathrm{ols}}}
\newcommand{\hatbbetamv}{\,\hat{\!\bbeta}^{\mathrm{mv}}}
\newcommand{\tildebbeta}{\,\tilde{\!\bbeta}}
\newcommand{\bbetaridge}{\bbeta^{\mathrm{ridge}}}
\newcommand{\bbetacoop}{\bbeta^{\mathrm{coop}}}
\newcommand{\bbetagroup}{\bbeta^{\mathrm{group}}}
\newcommand{\bbetaols}{\bbeta^{\mathrm{ols}}}

\newcommand{\hattheta}{\hat{\theta}}
\newcommand{\hatbtheta}{\,\hat{\!\btheta}}
\newcommand{\hatthetalasso}{\,\hat{\!\theta}^{\mathrm{lasso}}}
\newcommand{\hatthetacoop}{\,\hat{\!\theta}^{\mathrm{coop}}}
\newcommand{\hatthetagroup}{\,\hat{\!\theta}^{\mathrm{group}}}
\newcommand{\hatthetaspgroup}{\,\hat{\!\theta}^{\mathrm{spgroup}}}
\newcommand{\hatthetaridge}{\,\hat{\!\theta}^{\mathrm{ridge}}}
\newcommand{\hatthetaols}{\,\hat{\!\theta}^{\mathrm{ols}}}
\newcommand{\hatbthetacoop}{\,\hat{\!\btheta}^{\mathrm{coop}}}
\newcommand{\hatbthetagroup}{\,\hat{\!\btheta}^{\mathrm{group}}}
\newcommand{\hatbthetaridge}{\,\hat{\!\btheta}^{\mathrm{ridge}}}
\newcommand{\hatbthetalasso}{\,\hat{\!\btheta}^{\mathrm{lasso}}}
\newcommand{\hatbthetaols}{\,\hat{\!\btheta}^{\mathrm{ols}}}
\newcommand{\hatbthetamv}{\,\hat{\!\btheta}^{\mathrm{mv}}}
\newcommand{\tildebtheta}{\,\tilde{\!\btheta}}
\newcommand{\bthetaridge}{\btheta^{\mathrm{ridge}}}
\newcommand{\bthetacoop}{\btheta^{\mathrm{coop}}}
\newcommand{\bthetagroup}{\btheta^{\mathrm{group}}}
\newcommand{\bthetaols}{\btheta^{\mathrm{ols}}}

\newcommand{\bTheta}{\boldsymbol\Theta}

\newcommand{\transpose}[1]{\matr{#1}^\trans}
\newcommand{\crossprod}[2]{\transpose{#1} \matr{#2}}
\newcommand{\tcrossprod}[2]{\matr{#1} \transpose{#2}}

%\newcommand{\bbeta}{\boldsymbol\beta}
%\newcommand{\bPsi}{\boldsymbol\Psi}
\newcommand{\ba}{\mathbf{a}}
\newcommand{\bb}{\mathbf{b}}

\newcommand{\bmu}{\boldsymbol{\mu}}
\newcommand{\bSigma}{\boldsymbol{\Sigma}}
\newcommand{\bOmega}{\boldsymbol{\Omega}}
\newcommand{\bsigma}{\boldsymbol{\sigma}}
\newcommand{\bomega}{\boldsymbol{\omega}}

\newcommand{\tP}{\tilde{p}}
\newcommand{\diag}{\text{diag}}

\newcommand{\I}{\mathbf{1}}
\newcommand{\clG}{\mathcal{G}}
\newcommand{\clV}{\mathcal{V}}
\newcommand{\clE}{\mathcal{E}}
\newcommand{\clJ}{\mathcal{J}}
\newcommand{\clN}{\mathcal{N}}
\newcommand{\clP}{\mathcal{P}}
\newcommand{\clA}{\mathcal{A}}
\newcommand{\clD}{\mathcal{D}}
\newcommand{\clH}{\mathcal{H}}
\newcommand{\clO}{\mathcal{O}}
\newcommand{\clS}{\mathcal{S}}
\newcommand{\bW}{\mathbf{W}}
\newcommand{\bM}{\mathbf{M}}
\newcommand{\bR}{\mathbf{R}}
\newcommand{\bm}{\mathbf{m}}
\newcommand{\bs}{\mathbf{s}}
\newcommand{\bS}{\mathbf{S}}
\newcommand{\bO}{\mathbf{O}}
\newcommand{\bA}{\mathbf{A}}
\newcommand{\bE}{\mathbf{E}}
%\newcommand{\bI}{\mathbf{I}}
%\newcommand{\bP}{\mathbf{P}}
\newcommand{\bL}{\mathbf{L}}
\newcommand{\be}{\mathbf{e}}
\newcommand{\bg}{\mathbf{g}}
\newcommand{\bv}{\mathbf{v}}
\newcommand{\bu}{\mathbf{u}}
\newcommand{\bx}{\mathbf{x}}
\newcommand{\by}{\mathbf{y}}
%\newcommand{\bY}{\mathbf{Y}}
%\newcommand{\bz}{\mathbf{z}}
%\newcommand{\bX}{\mathbf{X}}
%\newcommand{\bzr}{\mathbf{0}}
\newcommand{\bH}{\mathbf{H}}
\newcommand{\bC}{\mathbf{C}}
\newcommand{\bB}{\mathbf{B}}
\newcommand{\bT}{\mathbf{T}}
\newcommand{\bV}{\mathbf{V}}
\newcommand{\bD}{\mathbf{D}}
\newcommand{\bU}{\mathbf{U}}
\newcommand{\bZ}{\mathbf{Z}}
%\newcommand{\balpha}{\boldsymbol\alpha}
%\newcommand{\bkappa}{\boldsymbol\kappa}
\newcommand{\bvarphi}{\boldsymbol\varphi}
\newcommand{\btheta}{\boldsymbol\theta}
\newcommand{\bgamma}{{\boldsymbol\gamma}}
%\newcommand{\bepsilon}{\boldsymbol\epsilon}
%\newcommand{\bvarepsilon}{\boldsymbol\varepsilon}
\newcommand{\blambda}{\boldsymbol\lambda}
\newcommand{\rsa}{\emphase{\mathversion{bold}{$\rightsquigarrow$}~}}


\newcommand\xylabellarge[4]{% #1 Image, #2=XText, #3=YText,, #4=Title,
  \setlength{\unitlength}{0.9\textwidth}%
  \begin{picture}(1,0.6)%
    \put(0.025,0.025){\includegraphics[width=0.95\unitlength]{#1}}
    \put(0.525,0){\makebox[0cm]{\small#2}}
    \put(0,0.3){\rotatebox{90.0}{\makebox[0cm]{\small#3}}}
    \put(0.525,0.6){\makebox[0cm]{#4}}
  \end{picture}%
}

\newcommand\xylabelsquare[4]{% #1 Image, #2=XText, #3=YText,, #4=Title,
  \setlength{\unitlength}{0.5\linewidth}%
  \begin{picture}(1,1)%
    \put(0.05,0.05){\includegraphics[width=\unitlength]{#1}}
    \put(0.575,-0.025){\makebox[0cm]{\small#2}}
    \put(-0.05,0.575){\rotatebox{90.0}{\makebox[0cm]{\small#3}}}
    \put(0.575,1.075){\makebox[0cm]{#4}}
  \end{picture}%
}

\newcommand\xylabel[3]{% #1 Image, #2=XText, #3=YText,
  \setlength{\unitlength}{0.18\textwidth}
  \begin{picture}(1,1)
    \put(0,0){\includegraphics[type=pdf,ext=.pdf,read=.pdf,width=0.18\textwidth]{#1}}
    \put(0.85,0.325){\small#2}
    \put(0.275,0.875){\small#3}
  \end{picture}
}

\newcommand\xyzlabel[4]{% #1 Image, #2=XText, #3=YText,, #3=ZText,
  \setlength{\unitlength}{0.18\textwidth}
  \begin{picture}(1,1)
    \put(0,0){\includegraphics[type=pdf,ext=.pdf,read=.pdf,width=0.18\textwidth]{#1}}
    \put(0.85,0.25){\small#2}
    \put(0.25,0.1){\small#3}
    \put(0.325,0.85){\small#4}
  \end{picture}
}



% Options
\makeatletter%%  
  % Linkfarbe, {0,0.35,0.35} für Türkis, {0,0,0} für Schwarz 
  \definecolor{linkcolor}{rgb}{0,0.35,0.35}
  % Zeilenabstand für bessere Leserlichkeit
  \def\mystretch{1.75} 
  % Publisher definieren
  \newcommand\publishers[1]{\newcommand\@publishers{#1}} 
  % Enumerate im 1. Level: \alph für a), b), \dotsc
  \renewcommand{\labelenumi}{\alph{enumi})} 
  % Enumerate im 2. Level: \roman für (i), (ii), \dotsc
  \renewcommand{\labelenumii}{(\roman{enumii})}
  % Zeileneinrückung am Anfang des Absatzes
  \setlength{\parindent}{0pt} 
  % Verweise auf Enumerate, z.B.: 3.2 a)
  \setlist[enumerate,1]{ref={\thesatz ~ \alph*)}}
  % Für das Proof-Environment: 'Beweis:' anstatt 'Beweis.'
  \xpatchcmd{\proof}{\@addpunct{.}}{\@addpunct{:}}{}{} 
  % Nummerierung der Bilder, z.B.: Abbildung 4.1
  \@ifundefined{thechapter}{}{\def\thefigure{\thechapter.\arabic{figure}}} 
\makeatother%

\title{Lectures in Game Thoery}
\author{Manish Krishan Lal}
\date{today ~\vspace{0.2cm} \\ Summer School 2019}
\publishers{ICTS-DCS}

%% Math. Definitions
%\newcommand{\C}{\mathbb{C}}
\newcommand{\N}{\mathbb{N}}
\newcommand{\Q}{\mathbb{Q}}
%\newcommand{\R}{\mathbb{R}}
\newcommand{\Z}{\mathbb{Z}}

%% Theorems (unnamedtheorem = Theorem ohne Namen)
\newtheoremstyle{named}{}{}{\normalfont}{}{\bfseries}{:}{0.25em}{#2 \thmnote{#3}}
\newtheoremstyle{itshape}{}{}{\itshape}{}{\bfseries}{:}{ }{}
\newtheoremstyle{normal}{}{}{\normalfont}{}{\bfseries}{:}{ }{}
\renewcommand*{\qed}{\hfill\ensuremath{\square}}

\theoremstyle{named}

\newtheorem*{beispiel*}{Beispiel}
\newtheorem*{example*}{example}
\newtheorem*{beispiele}{Beispiele}
\newtheorem*{examples}{Examples}
\newtheorem*{bemerkung}{Bemerkung} 
\newtheorem*{comment*}{Comment} 
\newtheorem*{bemerkungen}{Bemerkungen}
\newtheorem*{bezeichnung}{Bezeichnung}
\newtheorem*{eigenschaften}{Eigenschaften}
\newtheorem*{erinnerung}{Erinnerung}
\newtheorem*{folgerung*}{Folgerung}
\newtheorem*{folgerungen}{Folgerungen}
\newtheorem*{hilfssatz*}{Hilfssatz}
\newtheorem*{regeln}{Regeln}
\newtheorem*{schreibweise}{Schreibweise}
\newtheorem*{schreibweisen}{Schreibweisen}
\newtheorem*{uebung}{übung}
\newtheorem*{vereinbarung}{Vereinbarung}

%% Template
\makeatletter%
\DeclareUnicodeCharacter{00A0}{ } \pgfplotsset{compat=1.7} \hypersetup{colorlinks,breaklinks, urlcolor=linkcolor, linkcolor=linkcolor, pdftitle=\@title, pdfauthor=\@author, pdfsubject=\@title, pdfcreator=\@publishers}\DeclareOption*{\PassOptionsToClass{\CurrentOption}{report}} \ProcessOptions \def\baselinestretch{\mystretch} \setlength{\oddsidemargin}{0.125in} \setlength{\evensidemargin}{0.125in} \setlength{\topmargin}{0.5in} \setlength{\textwidth}{6.25in} \setlength{\textheight}{8in} \addtolength{\topmargin}{-\headheight} \addtolength{\topmargin}{-\headsep} \def\pulldownheader{ \addtolength{\topmargin}{\headheight} \addtolength{\topmargin}{\headsep} \addtolength{\textheight}{-\headheight} \addtolength{\textheight}{-\headsep} } \def\pullupfooter{ \addtolength{\textheight}{-\footskip} } \def\ps@headings{\let\@mkboth\markboth \def\@oddfoot{} \def\@evenfoot{} \def\@oddhead{\hbox {}\sl \rightmark \hfil \rm\thepage} \def\chaptermark##1{\markright {\uppercase{\ifnum \c@secnumdepth >\m@ne \@chapapp\ \thechapter. \ \fi ##1}}} \pulldownheader } \def\ps@myheadings{\let\@mkboth\@gobbletwo \def\@oddfoot{} \def\@evenfoot{} \def\sectionmark##1{} \def\subsectionmark##1{}  \def\@evenhead{\rm \thepage\hfil\sl\leftmark\hbox {}} \def\@oddhead{\hbox{}\sl\rightmark \hfil \rm\thepage} \pulldownheader }	\def\chapter{\cleardoublepage  \thispagestyle{plain} \global\@topnum\z@ \@afterindentfalse \secdef\@chapter\@schapter} \def\@makeschapterhead#1{ {\parindent \z@ \raggedright \normalfont \interlinepenalty\@M \Huge \bfseries  #1\par\nobreak \vskip 40\p@ }} \newcommand{\indexsection}{chapter} \patchcmd{\@makechapterhead}{\vspace*{50\p@}}{}{}{}
	% Titlepage
	\def\maketitle{ \begin{titlepage} 
			~\vspace{3cm} 
		\begin{center} {\Huge \@title} \end{center} 
	 		\vspace*{1cm} 
	 	\begin{center} {\large \@author} \end{center} 
	 	\begin{center} \@date \end{center} 
	 		\vspace*{7cm} 
	 	\begin{center} \@publishers \end{center} 
	 		\vfill 
	\end{titlepage} }
\makeatother%


